%\documentclass{slides}
%\documentclass[bw]{crocs-slide}
\documentclass[nologo]{slides}
%\documentclass[bw, nologo]{crocs-slide}

\usepackage[czech]{babel}

\title{Java 01}

\author[Lukáš Němec]{%
    \makebox[0.5\textwidth][l]{\textbf{Lukáš Němec}}\\%
    \vskip 1em
    %\makebox[0.5\textwidth][l]{Joe Bloggs}\\%
    \vskip 1em
    %\makebox[0.5\textwidth][l]{Tommy Atkins}%
}

\begin{document}

\maketitle

\begin{frame}
    \frametitle{Java je}

    \begin{itemize}
        \item jazyk třetí generace (stejně jako C, C#, Pascal ...)
        \item objektově orientovaný jazyk - OOP (stejně jako C++, C#, Delphi ...)
        \item jednodušší než C++, srovnatelná s C#
        \item Case sensitive - rozlišuje velikost písmen
    \end{itemize}

\end{frame}

\begin{frame}
    \frametitle{Rychlost}

    \begin{itemize}
        \item pomalejší než kompilované jazyky (C, C++)
        \item srovnatelná s ostatními Just in Time kompilovanými jazyky (C#)
        \item rychlejší než jazyky bez kompilátoru (Python, Ruby, PHP)
    \end{itemize}

\end{frame}

\begin{frame}
    \frametitle{JVM}

    \begin{itemize}
        \item umožňuje multiplatformní programy
        \item výstupem překladače je mezikód (bytecode), který je JVM zpracováván
        \item provádí překlad mezikódu do strojového kódu
        \item je součástí JRE
    \end{itemize}

\end{frame}

\begin{frame}
    \frametitle{JRE vs JDK}
    \framesubtitle{Java Runtime Environment a Java Developement Environment}
    \begin{itemize}
        \item JRE < JDK
        \item JRE umožňuje Javu spustit
        \item JDK umožňuje Javu vyvíjet
    \end{itemize}

\end{frame}

\begin{frame}
    \frametitle{Použití Javy}

    \begin{itemize}
        \item konzolové a grafické aplikace
        \item webové stránky
        \item aplikace pro telefony
        \item JavaCard
    \end{itemize}

\end{frame}

\begin{frame}
    \frametitle{O čem to bude}

    \begin{itemize}
        \item základy OOP
        \item konzolové aplikace
        \item základní prvky Javy a jak je správně používat
        \item nějaká matematika
        \item samostatný projekt?
        \item ???
    \end{itemize}

\end{frame}

\begin{frame}
    \frametitle{O čem to nebude (zatím)}

    \begin{itemize}
        \item Swing (grafické aplikace)
        \item webové aplikace
        \item aplikace pro telefony
        \item JavaCard
    \end{itemize}

\end{frame}

\begin{frame}
    \frametitle{Vývojová prostředí - IDE}

    \begin{itemize}
        \item NetBeans
        \item Eclipse
        \item IntelliJ IDEA
    \end{itemize}

Ostatní:
    \begin{itemize}
        \item Textový editor
        \item BlueJ
        \item ...
    \end{itemize}

\end{frame}

\begin{frame}
    \frametitle{LukeMcNemee.eu}

    \begin{itemize}
        \item sekce lužánky info - Java
        \item návod k Javě co nainstalovat a různé alternativy
        \item snad pravidelně obsah jednotlivých hodin
        \item snad řešení jednotlivých úloh z hodin
        \item většinou formou odkazů na GitHub
    \end{itemize}

\end{frame}



\begin{frame}
    \frametitle{Dotazník}


\end{frame}

\begin{frame}
    \frametitle{Dotazník}
    \framesubtitle{Programování}

    \begin{itemize}
        \item Pascal, C
        \item Dynamická alokace
        \item OOP
        \item SQL, XML
        \item Web (HTML, ...)
        \item grafické aplikace
    \end{itemize}

\end{frame}

\begin{frame}
    \frametitle{Dotazník}
    \framesubtitle{Matematika}

    \begin{itemize}
        \item obsahy, obvody atd. (kužel ..)
        \item prvočísla
        \item mocniny, faktoriál
        \item kombinatorika, pravděpodobnost
    \end{itemize}

\end{frame}

\begin{frame}
    \frametitle{Dotazník}
    \framesubtitle{Crypto}

    \begin{itemize}
        \item primitivní (Caesar, rot13 ...)
        \item velká prvočísla
        \item RSA, AES ...

    \end{itemize}

\end{frame}

\begin{frame}
    \frametitle{Dotazník}
    \framesubtitle{Ostatní}

Verzovací nástroje
    \begin{itemize}
        \item SVN
        \item Git
        \item jiné?

    \end{itemize}

Angličtina
    \begin{itemize}
        \item na úrovni Get, isEmpty, remove ...
        \item názvy funkcí, proměnných
        \item pochopit základ z JavaDoc

    \end{itemize}

\end{frame}


\begin{frame}
    \frametitle{Hello World} 


\end{frame}

\end{document}
